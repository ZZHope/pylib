% *********** Document name and reference:
% Title of document
\renewcommand{\ndoctitle}{mppnp.py: The python workhorse to get, plot and analyse se files} 
% Document category acronym 
\renewcommand{\ndocname}{mppnp}                      
% svn dir
\renewcommand{\svndir}{svn://forum.astro.keele.ac.uk/utils/pylib}  
% Contributors to this document
\renewcommand{\ndoccontribs}{DAC,FH,DM size that is used.}

%\chapter{\ndoctitle}   % uncomment this if compiled into book in docs directory
\ngtitle{\ndoctitle}     % uncomment this is compiled into chapter wrapper this directory


Document name: \ndocname \\
SVN directory: \svndir\\
Contributors: \ndoccontribs\\



{  \textbf{Abstract:} \slshape
mppnp gives access to data and methods needed to process data from se files, both the stellar evolution output files of type \texttt{.se.h5} as well as mppnp output, for example in H5\_out dir \texttt{.out.h5}, or in H5\_surf or H5\_restart. Like the other pylib modules it has access to the methods in the superclass Data\_Plot (see relevant chapter below). 
}
%##############################################################
%# Section: Introduction
%##############################################################

\section{Introduction}
\index{mppnp}
Welcome to the mppnp.py user Guide. The purpose of mppnp.py is to provide similiar capabilities to the ppn.py code but on a massively parallel scale. Mppnp.py provides similiar file methods such as sedir and sefiles and header and column attributes. 
A major caveat applied to using this class in the dramatic increase in processing time in dealing with  the increasing amount of data.
Please consult the docstrings in mppnp.py for more information.

This document will provide the user with a 
tutorial and walk through of the tools contained within mppnp.py and
how a typical user would go about using and working with this python module.
\subsection{Tutorial and Walkthough}

Currently,  
in this instance instead of pre-generated textual .DAT files we are using hdf5 files.  
In order to read the hdf5 files one must have the h5py module installed or added to your PYTHONPATH.
\newline

In bash:

\begin{verbatim}
export PYTHON=/location/of/additional/pythonfiles
\end{verbatim}

In tcsh:

\begin{verbatim}
setenv PYTHON /location/of/additional/pythonfiles
\end{verbatim}


Furthermore, in order to manipulate the data into plots you need to locate your ipython executable.
If the ipython executable is {\bf not} currently  located in your PATH variable a
\newline
\begin{verbatim}
 whereis ipython
\end{verbatim}
will provide you with the absolute location of the ipython executable.  


\subsection{Getting Started}

To view an example
\begin{verbatim}
>import mppn as mp
\end{verbatim}

Next, initiate a se class instance with either a multiple file set.
\begin{verbatim}
pt=mp.se('/path/to/dir/with/se-files')
\end{verbatim}

or initiate a se class instance with a single file.
\begin{verbatim}
pt=mp.se('/path/to/dir/','particular_se-files')
\end{verbatim}


The particular header and columns attributes of this instance can be determined
be the following methods.

\begin{verbatim}
pt.se.cattrs
pt.se.hattrs
pt.se.dcols
\end{verbatim}

Based on the data within your file. The following methods may return the following.
\begin{verbatim}
pt.se.dcols
>>['convection_indicator', 'dcoeff', 'mass', 'pressure', 'radius',
 'rho', 'temperature', 'velocity', 'x c', 'xh', 'xhe', 'xmg', 'xne',
 'xne22', 'xni', 'xo', 'xsi']
\end{verbatim}



\begin{verbatim}
pt.se.cattrs
>>['TITLE', 'CLASS', 'VERSION', 'model_number', 'total_mass', 'age',
 'deltat', 'shellnb', 'logL', 'logT eff']
\end{verbatim}


Using the get method in this class with return a data array based on either the variables obtained
by the previous pt.se.cattrs or pt.se.dcol method.

\begin{verbatim}
temparray = pt.get("temperature")
\end{verbatim}

The length of this temparray will be based on the length of the following method
since each cycle will have a temperature variable.

\begin{verbatim}
pt.se.cycles
\end{verbatim}

In order to minimize processing time in plotting a very large dataset one could use the python range function in order to iterate over a subsequence of numbers.


\begin{verbatim}
smaller_set=range(lownumber,highnumber,increment)
\end{verbatim}







\subsection{Disclaimer}

		
\subsection{History} 
This document history complements the svn log.

\begin{tabular*}{\textwidth}{lll}
\hline
Authors & yymmdd & Comment \\
\hline
FH & 110702 & generate template \\
\hline
DM & 120424 & generate basic doc \\
\hline
\end{tabular*}


\subsection{Contact}
If any bugs do appear or if there are any questions, please email hoshi@uvic.ca
% --------------- latex template below ---------------------------
\begin{verbatim}

\end{verbatim}


%\begin{figure}[htbp]
%   \centering
%%   \includegraphics[width=\textwidth]{layers.jpg} % 
%      \caption{}   \includegraphics[width=0.48\textwidth]{FIGURES/HRD90ms.png}  
%   \includegraphics[width=0.48\textwidth]{FIGURES/HRD150ms.png}  
%
%   \label{fig:one}
%\end{figure}
%
%\begin{equation}
%Y\_a = Y\_k + \sum\_{i \neq k} Y\_i
%\end{equation}
%
%{
%%\color{ForestGreen}
%\sffamily 
%  {\center  --------------- \hfill {\bf START: Some special text} \hfill ---------------}\\
%$Y\_c$ does not contain ZZZ but we may assign one $Y_n$ to XYZ which is the decay product of the unstable nitrogen isotope JJHJ. %
%
%{\center ---------------  \hfill {\bf END:Some special text} \hfill ---------------}\\
%}
