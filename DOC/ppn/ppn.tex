% *********** Document name and reference:
% Title of document
\renewcommand{\ndoctitle}{ppn.py: Simple ppn Data Structure} 
% Document category acronym 
\renewcommand{\ndocname}{ppm}                      
% svn dir
\renewcommand{\svndir}{svn://forum.astro.keele.ac.uk/utils/pylib}  
% Contributors to this document
\renewcommand{\ndoccontribs}{DAC}

%\chapter{\ndoctitle}   % uncomment this if compiled into book in docs directory
\ngtitle{\ndoctitle}     % uncomment this is compiled into chapter wrapper this directory


Document name: \ndocname \\
SVN directory: \svndir\\
Contributors: \ndoccontribs\\



{  \textbf{Abstract:} \slshape
A simple lession and tuitorial on using ppn.py in an effective and efficient manner
}
%##############################################################
%# Section: Introduction
%##############################################################

\section{Introduction}
\index{ppn}Welcome to the ppn.py User Guide.  The purpose of ppn.py is
to provide a simple data structure to help the user work with simple data files. 
ppn.py contains the classes xtime and abu\_vector in which we work with the required data.
xtime works with the xtime.dat file (by default) and abu\_vector with iso\_massf files (by default)

This document will provide the user with a 
tutorial and walk through of the tools contained within ppn.py and
how a typical user would go about using and working with this python module.

\subsection{Disclaimer}
Programmers have a bad habbit of not updating documentation, espically then it is in  a .tex file.
It is then recommeded that the user look at the docstring contained within the pyhton script, there the information is more likely to be accurite up to date
, more about accessing that later
\subsection{Nature of iso\_massf files}
There are a couple key identifiers to iso\_massf files \newline
The first non white space character in a header line is a \#.\newline
Header attributes are separated from their value by white space or by white space 
	surrounding an equals sign.\newline
An Header attribute is separated by the previous Header attribute by white space
	or a line break.\newline
There are only 6 data columns. The first being the number, second being Z, third being A, Fourth isomere state
	A fifth abundance\_yps and finally the element name.\newline
The first Five columns consist purely of numbers, no strings are allowed.\newline 
Of the values in the final column, the name column, the first two are letters \newline
specifying the element name, and the rest are spaces or numbers (in that strict
order), except for the element names: Neut and Prot\newline
All the profile files in the directory have the same cycle attributes.	\newline
The cycle numbers of the 'filename'+xxxxx start at 0 .\newline
No cycle numbers are skipped, ie if cycle 0 and 3 are present, 1 and 2 must be here aswell.\newline
PPN files allways end in .DAT and are not allowed any '.'\newline
The can not be any blank lines in the data files.\newline

\section{Tutorial and Walkthough}

\subsection{Downloading and configuring}
Before the user can start working with ppn.py, they first must download the python 
module from svn:/ /forum.astro.keele.ac.uk/utils/pylib.  Also the user must download the python module data\_plot.py and utils.py. Once the package is placed in the filesystem, the user must add the location
of ppn.py to their python path.  What this means is that the user is telling python where to look when importing modules.
To do this:
\newline
A) Navigate to the location of ppm.py in a terminal and type,

\begin{verbatim}
pwd
\end{verbatim}

What should output is the location of ppm.py.
\newline
B) In a unix environment, open up your .bashrc file that is usually located in your home directory.  If it is not there, do not worry, it is a hidden file.
in the users home directory and open it.  If it does not exist, create it using a text editor.
Now in this .bashrc there may or may not be a line like:

\begin{verbatim}
export PYTHONPATH=/directory/:
\end{verbatim}

If there is add the location of ppm.py to the end of the line and a semi-colon.  It now should look like:

\begin{verbatim}
export PYTHONPATH=/directory/:/location/of/ppn.py:
\end{verbatim}

If that line did not already exist add this line to the .bashrc file.

\begin{verbatim}
export PYTHONPATH=$PYTHONPATH:/location/of/ppn.py:
\end{verbatim}

Now we must repeat these steps with data\_plot.py and utils.py unless if they are in the same directory as ppm.py, in which case everything is ready to go.


\subsection{Getting Started}
To view an example runthrough of the classes xtime and abuvector,
first in our python command line (accessed by typing the command 'python' in your terminal) or in a python script,
type the command:
\begin{verbatim}
>import ppn as pn
\end{verbatim}
Now, type either help(pn.xtime) and then help(pn.abu\_vector) to find an example in the docstring.

\subsection{Methods And Functions}
Here, all the methods (public ones that is) that are available to xtime and abu\_vector. To view more information 
about any of them please consult the docstring by typeing help(pn.method), assuming
the user imported this module as pn.\newline
\newline
{\bf First for class xtime}\newline
{\bf get()}\newline 
Get one data column with the data.\newline
{\bf plot\_xtime()}\newline 
Make a simple plot of two columns against each other.\newline
{\bf plot()}\newline 
Simple plot function.\newline
{\bf getplotMulti()}\newline 
Method for plotting multiple plots and saving it to multiple pngs.\newline
\newline
{\bf Now for class abu\_vector}\newline
{\bf get()}\newline 
In this method a column of data for the associated attribute is
returned. If fname is a list or None a list of each cycles in 
fname or all cycles is returned\newline
{\bf getElement()}\newline 
In this method instead of getting a particular column of data,
the program gets a paticular row of data for a paticular 
element name.\newline
{\bf getCycleData()}\newline 
In this method a column of data for the associated cycle attribute is returned\newline
{\bf getColData()}\newline 
In this method a column of data for the associated column attribute is returned\newline
{\bf plot()}\newline 
Simple plot function.\newline
{\bf getplotMulti()}\newline 
Method for plotting multiple plots and saving it to multiple pngs.\newline
{\bf clear()}\newline 
Method for removing the title and/or xlabel and/or Ylabel \newline
{\bf abu\_chart()}\newline 
Plots an abundence chart\newline
{\bf abu\_chartMulti()}\newline 
Method that plots abundence charts and saves those figures to a .png file (by default). Plots a figure for each cycle in the argument cycle\newline
{\bf iso\_abundMulti()}\newline 
Method that plots isotope abundence plots and saves those figures to a .png file (by default). Plots a figure for each cycle in the argument cycle\newline
{\bf iso\_abund()}\newline 
Plots the abundance of all the chemical isotopes
{\bf get()}\newline 
		
\subsection{History} 
This document history complements the svn log.

\begin{tabular*}{\textwidth}{lll}
\hline
Authors & yymmdd & Comment \\
\hline
DAC & 101112 & generate template \\
DAC & 101220 & Added documentation \\
\hline
\end{tabular*}


\subsection{Contact}
If any bugs do appear or if there are any questions, please email hoshi@uvic.ca
% --------------- latex template below ---------------------------
\begin{verbatim}

\end{verbatim}


%\begin{figure}[htbp]
%   \centering
%%   \includegraphics[width=\textwidth]{layers.jpg} % 
%      \caption{}   \includegraphics[width=0.48\textwidth]{FIGURES/HRD90ms.png}  
%   \includegraphics[width=0.48\textwidth]{FIGURES/HRD150ms.png}  
%
%   \label{fig:one}
%\end{figure}
%
%\begin{equation}
%Y\_a = Y\_k + \sum\_{i \neq k} Y\_i
%\end{equation}
%
%{
%%\color{ForestGreen}
%\sffamily 
%  {\center  --------------- \hfill {\bf START: Some special text} \hfill ---------------}\\
%$Y\_c$ does not contain ZZZ but we may assign one $Y_n$ to XYZ which is the decay product of the unstable nitrogen isotope JJHJ. %
%
%{\center ---------------  \hfill {\bf END:Some special text} \hfill ---------------}\\
%}
