% *********** Document name and reference:
% Title of document
\renewcommand{\ndoctitle}{ascii\_table.py: Simple Ascii Table Data Structure} 
% Document category acronym 
\renewcommand{\ndocname}{ppm}                      
% svn dir
\renewcommand{\svndir}{svn://forum.astro.keele.ac.uk/utils/pylib}  
% Contributors to this document
\renewcommand{\ndoccontribs}{DAC}

\input{chap}

Document name: \ndocname \\
SVN directory: \svndir\\
Contributors: \ndoccontribs\\



{  \textbf{Abstract:} \slshape
A simple lession and tuitorial on using ascii\_table.py in an effective and efficient manner
}
%##############################################################
%# Section: Introduction
%##############################################################

\section{Introduction}
\index{ascii\_table}Welcome to the Ascii Table User Guide.  The purpose of ascii\_table.py is
to provide a simple data structure to help the user work with simple data files. 
PPM.py contains the class YProfile in which we work with the required data.
This document will provide the user with a 
tutorial and walk through of the tools contained within ascii\_table.py and
how a typical user would go about using and working with this python module.
\subsection{Nature of Ascii Files}
There are a couple key identifiers to ascii table files \newline
A)Lines of header attributes begin with a capital H \newline
B)The next line after the header lines, is a line of column attribute names.\newline
C)The column attribute names are separated by '  ' by default or whatever the 
user dictates to the class by specifying sep.\newline
D)All the data columns are of equal length.\newline
E)Data columns are seperated by blank spaces.\newline
\section{Tutorial and Walkthough}

\subsection{Downloading and configuring}
Before the user can start working with ppm.py, they first must download the python 
module from svn:/ /forum.astro.keele.ac.uk/utils/pylib.  Also the user must download the python module data\_plot.py and utils.py. Once the package is placed in the filesystem, the user must add the location
of ppm.py to their python path.  What this means is that the user is telling python where to look when importing modules.
To do this:
\newline
A) Navigate to the location of ppm.py in a terminal and type,

\begin{verbatim}
pwd
\end{verbatim}

What should output is the location of ppm.py.
\newline
B) In a unix environment, open up your .bashrc file that is usually located in your home directory.  If it is not there, do not worry, it is a hidden file.
in the users home directory and open it.  If it does not exist, create it using a text editor.
Now in this .bashrc there may or may not be a line like:

\begin{verbatim}
export PYTHONPATH=/directory/:
\end{verbatim}

Now add the location of ppm.py to the end of the line and a semi-colon.  It now should look like:

\begin{verbatim}
export PYTHONPATH=/directory/:/location/of/ascii_table.py:
\end{verbatim}

If that line did not already exist add this line to the .bashrc file.

\begin{verbatim}
export PYTHONPATH=$PYTHONPATH:/location/of/ascii_table.py:
\end{verbatim}

Now we must repeat these steps with data\_plot.py and utils.py unless if they are in the same directory as ppm.py, in which case everything is ready to go.


\section{Getting Started}
Here I will go about explaining how the user would go about using ascii\_table.py for his or her purposes.
First in our python command line (accessed by typing the command 'python' in your terminal) or in a python script,
type the command:
\begin{verbatim}
>import ascii_table as a
\end{verbatim}
Now, baring any errors ppm should be loaded into your command line or script, and the user can begin
to work with ppm and its YProfile class. To create an instance of YProfile type:
\begin{verbatim}
>p=a.AsciiTable('file','Directory')
\end{verbatim}
Where 'file' should be a ascii table compatable file and 'Directory' should be the directory where the YProfile files, that the user wants
to work with, are located. If the user does not specify a directory by calling the class
like: 
\begin{verbatim}
>p=a.AsciiTable('file')
\end{verbatim}
The class will then default to the working directory.\newline
There also is another agument, called sep. This is the seperator that seperates the 
cycle attribute names. By default it equals '  '. For Example
\begin{verbatim}
>p=a.AsciiTable('file',sep=';')
\end{verbatim}
An actual, real life, example would be
\begin{verbatim}
>p=a.AsciiTable('C12.dat','./Run1/', sep=':')
\end{verbatim}

Now the user can use their AsciiTable instance as they see fit.

\section{Viewing Data}
Once the user has instantiated their instance, they are ready to start working with it.
One quick note, for the purposes of simplicity I will assume that the user has named
their instance, p.
\newline
To view the header attributes type:
\begin{verbatim}
>p.hattrs
\end{verbatim}
To view the Data Column attributes type
\begin{verbatim}
>p.dcols
\end{verbatim}

Now let us look at how to acquire data.
\newline
The basic way to acquire data is by the get function. The get method will dynamically determine what kind of attribute the user has passed in
be it a header cycle or column attribute.  It returns a numpy array.  It is called like:
\begin{verbatim}
data=p.get('attribute')
\end{verbatim}
Where 'attribute' is what data the user wants.
An actual, real life, example would be
\begin{verbatim}
>p.get('upper')
\end{verbatim}

\section{Plotting Data}
To plot data, this guide recommends the user to download and install ipython, it will let the user to use plotting more dynamically. To install it go do http://ipython.scipy.org/moin/ and follow the links.
But before the user does this, check if your system has ipython already installed by typing ipython in a terminal.
\newline
To run ipython
\begin{verbatim}
ipython --pylab --q4thread
\end{verbatim}
(note: the argument --q4thread is a work around for a bug in ipyhthon 0.10.1, if a different version is in use, this may not be needed)
\newline
Now import ppm and create an instance as was documented in the Getting Started section.
For AsciiTable, the DataPlot class in  data\_plot.py provides us with two methods plot() and clear().
\newline
To use the plot method
\begin{verbatim}
p.plot(attriX, attriY)
\end{verbatim}
Where attriX is the name of the attribute of what the user wants on the X axis and attriY is the name of the attribute of what the user wants on the Y axis.
This method automatically generates the title, the legend and the X and Y axis lables.

\section{Methods And Functions}
Here, all the methods (public ones that is) that are available to ppn.py will be explained in detail.
\newline\newline
{\bf get(attri).}
\newline  

Method that dynamically determines the type of attribute that is passed into this method. Also it then returns that attribute's
associated data.\newline
Input:\newline
attri: The attribute we are looking for.\newline
		
\section{History} 
This document history complements the svn log.

\begin{tabular*}{\textwidth}{lll}
\hline
Authors & yymmdd & Comment \\
\hline
DAC & 101117 & generate template and imput data\\

\hline
\end{tabular*}


\section{Contact}
If any bugs do appear or if there are any questions, please email hoshi@uvic.ca
% --------------- latex template below ---------------------------
\begin{verbatim}

\end{verbatim}


%\begin{figure}[htbp]
%   \centering
%%   \includegraphics[width=\textwidth]{layers.jpg} % 
%      \caption{}   \includegraphics[width=0.48\textwidth]{FIGURES/HRD90ms.png}  
%   \includegraphics[width=0.48\textwidth]{FIGURES/HRD150ms.png}  
%
%   \label{fig:one}
%\end{figure}
%
%\begin{equation}
%Y\_a = Y\_k + \sum\_{i \neq k} Y\_i
%\end{equation}
%
%{
%%\color{ForestGreen}
%\sffamily 
%  {\center  --------------- \hfill {\bf START: Some special text} \hfill ---------------}\\
%$Y\_c$ does not contain ZZZ but we may assign one $Y_n$ to XYZ which is the decay product of the unstable nitrogen isotope JJHJ. %
%
%{\center ---------------  \hfill {\bf END:Some special text} \hfill ---------------}\\
%}
