% *********** Document name and reference:
% Title of document
\renewcommand{\ndoctitle}{ppm.py: The YProfile Data Structure Tool} 
% Document category acronym 
\renewcommand{\ndocname}{ppm}                      
% svn dir
\renewcommand{\svndir}{svn://forum.astro.keele.ac.uk/utils/pylib}  
% Contributors to this document
\renewcommand{\ndoccontribs}{DAC}

%\chapter{\ndoctitle}   % uncomment this if compiled into book in docs directory
\ngtitle{\ndoctitle}     % uncomment this is compiled into chapter wrapper this directory


Document name: \ndocname \\
SVN directory: \svndir\\
Contributors: \ndoccontribs\\



{  \textbf{Abstract:} \slshape
A simple lession and tuitorial on using ppm.py in an effective and efficient manner
}
%##############################################################
%# Section: Introduction
%##############################################################

\section{Introduction}
\index{ppm}Welcome to the PPM User Guide.  The purpose of PPM.py is
to provide a simple data structure to help the user work with YProfile files. 
PPM.py contains the class YProfile in which we work with the required data.
This document will provide the user with a 
tutorial and walk through of the tools contained within ppm.py and
how a typical user would go about using and working with this python module. 
\section{Tutorial and Walkthough}

\subsection{Downloading and configuring}
Before the user can start working with ppm.py, they first must download the python 
module from svn:/ /forum.astro.keele.ac.uk/utils/pylib.  Also the user must download the python module data\_plot.py and utils.py. Once the package is placed in the filesystem, the user must add the location
of ppm.py to their python path.  What this means is that the user is telling python where to look when importing modules.
To do this:
\newline
A) Navigate to the location of ppm.py in a terminal and type,

\begin{verbatim}
pwd
\end{verbatim}

What should output is the location of ppm.py.
\newline
B) In a unix environment, open up your .bashrc file that is usually located in your home directory.  If it is not there, do not worry, it is a hidden file.
in the users home directory and open it.  If it does not exist, create it using a text editor.
Now in this .bashrc there may or may not be a line like:

\begin{verbatim}
export PYTHONPATH=/directory/:
\end{verbatim}

If there is add the location of ppm.py to the end of the line and a semi-colon.  It now should look like:

\begin{verbatim}
export PYTHONPATH=/directory/:/location/of/ppm.py:
\end{verbatim}

If that line did not already exist add this line to the .bashrc file.

\begin{verbatim}
export PYTHONPATH=$PYTHONPATH:/location/of/ppm.py:
\end{verbatim}

Now we must repeat these steps with data\_plot.py and utils.py unless if they are in the same directory as ppm.py, in which case everything is ready to go.


\subsection{Getting Started}
Here I will go about explaining how the user would go about using ppm.py for his or her purposes.
First in our python command line (accessed by typing the command 'python' in your terminal) or in a python script,
type the command:
\begin{verbatim}
>import ppm as pm
\end{verbatim}
Now, baring any errors ppm should be loaded into your command line or script, and the user can begin
to work with ppm and its YProfile class. To create an instance of YProfile type:
\begin{verbatim}
>p=pm.YProfile('DIRECTORY')
\end{verbatim}
'DIRECTORY' should be the directory where the YProfile files, that the user wants
to work with, are located. If the user does not specify a directory by calling the class
like: 
\begin{verbatim}
>pm.YProfile()
\end{verbatim}
the class will then default to the working directory.
Now the user should be presented with some feedback, the number of YProfiles in the selected 
directory,the range of NDumps in the directory and the range of times in the directory. For example:
\begin{verbatim}
There are 150 YProfile files in the . directory.
Ndump values range from 0 to 149
Time values range from 0.0 to 8962.31
\end{verbatim}
Now the user can use their YProfile instance as they see fit.

\subsection{Viewing Data}
Once the user has instantiated their instance, they are ready to start working with it.
One quick note, for the purposes of simplicity I will assume that the user has named
their instance, p.
\newline
To view the files that are available:
\begin{verbatim}
>p.files
\end{verbatim}
To view the header attributes type:
\begin{verbatim}
>p.hattrs.keys()
\end{verbatim}


To view the Cycle attributes type
\begin{verbatim}
>p.cattrs
\end{verbatim}
To view the Data Column attributes type
\begin{verbatim}
>p.dcols
\end{verbatim}

Now let us look at how to acquire data.
\newline
The basic way to acquire data is by the get function. The get method will dynamically determine what kind of attribute the user has passed in
be it a header cycle or column attribute.  It returns a numpy array or a list of numpy arrays(if resolution ='a', more on that later).  In the most general way, it is called like:
\begin{verbatim}
data=p.get('attribute', cycleNumber)
\end{verbatim}
Where 'attribute' is what data the user wants, and cycleNumber is which cycle this data will be read from.  Ie if cycleNumber was 20, YProfile-01-0020.bobaaa would be read.
This will satisfy most users needs, but if more detail is needed, please read further in the Methods And Functions subsection

\subsection{Plotting Data}
To plot data, this guide recommends the user to download and install ipython, it will let the user to use plotting more dynamically. To install it go do http://ipython.scipy.org/moin/ and follow the links.
But before the user does this, check if your system has ipython already installed by typing ipython in a terminal.
\newline
To run ipython
\begin{verbatim}
ipython --pylab --q4thread
\end{verbatim}
(note: the argument --q4thread is a work around for a bug in ipyhthon 0.10.1, if a different version is in use, this may not be needed)
\newline
Now import ppm and create an instance as was documented in the Getting Started section.
For YProfile, the DataPlot class in  data\_plot.py provides us with two methods plot() and clear().
\newline
To use the plot method
\begin{verbatim}
p.plot(attriX, attriY)
\end{verbatim}
Where attriX is the name of the attribute of what the user wants on the X axis and attriY is the name of the attribute of what the user wants on the Y axis.
This method automatically generates the title, the legend and the X and Y axis lables.

\subsection{Methods And Functions}
Here, all the methods (public ones that is) that are available to ppn.py will be explained in detail.  Warning Here be Dragons!
\newline
{\bf getHattrs()}
Returns a list of the header attribute names in sorted order. Same effect as the commands:
\begin{verbatim}
h=instance.hattrs.sorted()
h.sort()
\end{verbatim}
{\bf getDCols()}
Returns a list of the column attribute names. Same effect as the command:
\begin{verbatim}
instance.dcols
\end{verbatim}
{\bf getCattrs()}
Returns a list of the cycle attribute names. Same effect as the command:
\begin{verbatim}
instance.cattrs
\end{verbatim}

{\bf get(attri, FName=None,numType='ndump',resolution='H').}
\newline  
Method that dynamically determines the type of attribute that is 
passed into this method. Also it then returns that attribute's 
associated data.
The arguments are attri, fname, numType and rersolution.
Input: \newline
attri: The attribute we are looking for.\newline
Fname: Be the filename, Ndump or time, Defaults to the last NDump\newline
numType: designates how this function acts and how it interprets FName
defaults to file\newline
if numType is 'file', this function will get the desired attribute from
that file\newline
if numType is 'NDump' function will look at the cycle with that nDump\newline
if numType is 'T' or 'time' function will find the \_cycle with the 
closest time stamp\newline
Resolution: Data you want from a file, for example if the file 
contains two different sized columns of data for one attribute, the 
argument 'a' will return them all,
'h' will return the largest,
l will return the lowest
Default resolution is h\newline


{\bf getHeaderData(self,attri)}
\newline 
Returns a String or int of Header data that is associated with the attri
\newline
Input: \newline attri: the name of the attribute.\newline


{\bf getCycleData( attri,FName=None,numType='ndump',Single=False,resolution='H')}
\newline 
Returns a list (Datalist) of values for the given attribute or a single attribute in the file FName.
\newline
		Input:\newline
		attri: What we are looking for\newline
		Fname: Be the filename, Ndump or time, Defaults to the last NDump\newline
		numType: designates how this function acts and how it interprets FName
			 Defaults to file\newline
		if numType is 'file', this function will get the desired attribute from that file\newline
		if numType is 'NDump' function will look at the cycle with that nDump\newline
		if numType is 'T' or 'time' function will find the \_cycle with the closest time stamp \newline
		Single: A Boolean determining whether the user wants just the Attri contained
		        in the specified ndump or all the dumps below that ndump. 
		        Defaults to False\newline
		Resolution: Data you want from a file, for example if the file 
			    contains two different sized columns of data for one attribute, 
			    the argument 'a' will return them all, 'h' will return the largest, 
			    l will return the lowest
			    Default resolution is h\newline
			    
			    
{\bf getColData(attri, FName, numType='ndump' ,resolution='H', cycle=False)}
\newline 
Returns a list (Datalist) of values for the given attribute 
\newline
		Input: \newline
		numType: designates how this function acts and how it interprets FName
			 Defaults to file\newline
		if numType is 'file', this function will get the desired attribute from that file\newline
		if numType is 'NDump' function will look at the cycle with that nDump\newline
		if numType is 'T' or 'time' function will find the \_cycle with the closest time stamp \newline
		Fname: The name of the file we are looking or Ndump or time\newline
		attri: Attri is the attribute we are loking for\newline
		Resolution: Data you want from a file, for example if the file 
			    contains two different sized columns of data for one attribute, 
			    the argument 'a' will return them all, 'h' will return the largest, 
			    l will return the lowest
			    Default resolution is h\newline
		
\subsection{History} 
This document history complements the svn log.

\begin{tabular*}{\textwidth}{lll}
\hline
Authors & yymmdd & Comment \\
\hline
DAC & 101101 & generate template \\
DAC & 101112 & First final documentation done \\
\hline
\end{tabular*}


\subsection{Contact}
If any bugs do appear or if there are any questions, please email hoshi@uvic.ca
% --------------- latex template below ---------------------------
\begin{verbatim}

\end{verbatim}


%\begin{figure}[htbp]
%   \centering
%%   \includegraphics[width=\textwidth]{layers.jpg} % 
%      \caption{}   \includegraphics[width=0.48\textwidth]{FIGURES/HRD90ms.png}  
%   \includegraphics[width=0.48\textwidth]{FIGURES/HRD150ms.png}  
%
%   \label{fig:one}
%\end{figure}
%
%\begin{equation}
%Y\_a = Y\_k + \sum\_{i \neq k} Y\_i
%\end{equation}
%
%{
%%\color{ForestGreen}
%\sffamily 
%  {\center  --------------- \hfill {\bf START: Some special text} \hfill ---------------}\\
%$Y\_c$ does not contain ZZZ but we may assign one $Y_n$ to XYZ which is the decay product of the unstable nitrogen isotope JJHJ. %
%
%{\center ---------------  \hfill {\bf END:Some special text} \hfill ---------------}\\
%}
