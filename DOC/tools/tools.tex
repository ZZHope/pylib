% *********** Document name and reference:
% Title of document
\renewcommand{\ndoctitle}{data\_plot.py and utils.py: A guide to using these superclasses} 
% Document category acronym 
\renewcommand{\ndocname}{Tools}                      
% svn dir
\renewcommand{\svndir}{svn://forum.astro.keele.ac.uk/utils/pylib/DOC/tools}  
% Contributors to this document
\renewcommand{\ndoccontribs}{DAC}

%\chapter{\ndoctitle}   % uncomment this if compiled into book in docs directory
\ngtitle{\ndoctitle}     % uncomment this is compiled into chapter wrapper this directory


Document name: \ndocname \\
SVN directory: \svndir\\
Contributors: \ndoccontribs\\



{\textbf{Abstract:} \slshape
This Document will allow a user to understand the methods contained within the super classes
data\_plot.py and utils.py.  Also it will describe how to extend any module with 
these super classes.
}
%##############################################################
%# Section: Introduction
%##############################################################

\section{Introduction}
\index{data\_plot}\index{utils}Welcome to the Tools User Guide.  The purpose of these tools is to provide a structure and inheritance to 
the suite of python modules. Currently this suite contains mesa.py, ppm.py, ppn.py, mppnp.py, ascii\_table.py, as well as the superclasses data\_plot.py and utils.py
The goal behind the creation of these superclasses was that these modules in this suite all had related, general functions, for example they all had a plot function.
It was decided that these general methods and functions would be contained in these superclasses.
This way, we would have a greater and more concrete hierarchy to out software system.
This document will show the functions and variables contained within these superclasses and how a user could extend their module with data\_plot.py and utils.py

\section{Downloading and configuring}
Download the module from from svn://forum.astro.keele.ac.uk/utils/pylib/.   Once the package is placed in the filesystem, the user must add the location
of data\_plot.py and utils.py to their python path.

\section{Functions, Methods and Variables}
Here the document will go about explaining how the user would go about using the functions, methods and variables
contained within data\_plot.py and utils.py.

\subsection{data\_plot.py}
The class DataPlot contained within data\_plot.py is a super class that contains functions that pertain to plotting data.
\newline
\begin{verbatim}
plot(self,atriX,atriY, FName=None,numType='ndump',legend=None,
    labelX=None, labelY=None,indexX=None, indexY=None, title=None, 
    shape='.',logX=False, logY=False, base=10)
\end{verbatim}	
Simple function that plots atriY as a function of atriX. This method will automatically find the data by calling
get(attribute,FName, numType,resolution) for a YProfile class and get(attribute) for everything else, and plot the requested data.\newline
		Input:\newline
		atriX: The name of the attribute you want on the x axis \newline
		atriY: The name of the attribute you want on the Y axis \newline
		Fname: Be the filename, Ndump or time, Defaults to the last 
			NDump, only need for a YProfile class.\newline
		numType: designates how this function acts and how it interprets 
			FName Defaults to file, only need for a YProfile class.\newline
		if numType is 'file', this function will get the desird attribute from that file\newline
		if numType is 'NDump' function will look at the cycle with that nDump\newline
		if numType is 't' or 'time' function will find the \_cycle with the closest time stamp \newline
		labelX: The label on the X axis \newline
		labelY: The label on the Y axis \newline
		indexX: Depreciated: If the get method returns a list of lists, indexX
			would be the list at the index indexX in the list.\newline
		indexY: Depreciated: If the get method returns a list of lists, indexY
			would be the list at the index indexX in the list.\newline
		shape: What shape and colour the user would like their plot in.
		       \newline
		title: The Title of the Graph \newline
		logX: A boolean of weather the user wants the x axis logarithmically\newline
		logY: A boolean of weather the user wants the Y axis logarithmically\newline
		base: The base of the logarithm. Default = 10\newline
		WARNING: Unstable if get returns a list with only one element (x=[0])\newline
An example of calling this method would be:
\begin{verbatim}
instance.plot('A','FVconv')
\end{verbatim}

\begin{verbatim}
plot_prof_1(self,species,keystring,xlim1,xlim2,ylim1,ylim2)
\end{verbatim}
Plot one species for cycle between xlim1 and xlim2. This plot function if for the classes mesa\_profile and h5TPlotTools in mppnp.py.\newline		
	species      - which species to plot \newline
	keystring    - label that appears in the plot, if the subclass is h5TPlotTools this gets interpereted as the model number \newline
	xlim1, xlim2 - mass coordinate range\newline                                                 
	ylim1, ylim2 - mass fraction coordinate range\newline


\begin{verbatim}
classTest()
\end{verbatim}
Determines what the type of class instance the subclass is, so
		we can dynically determine the behaviour of methods.
		
		This method NEEDS to be modified if any names of files or classes
		are changed
\newline
\newline
\begin{verbatim}
logarithm(self,tmpX,tmpY,logX,logY,base)
\end{verbatim}
Logarithm takes in two lists of data and then take the 
		logarithm of each element with the base. If there is an
		element at some index that is less than or equal to 0, this 
		method will remove the elements in both lists at that index\newline
		input:\newline
		tmpX: a list of data\newline
		tmpY: a list of data\newline
		logX: If the user wants tmpX loged\newline
		logY: If the user wants tmpY loged\newline
		base: The base of the logarithm, defaults to 10\newline
		
		
\begin{verbatim}
clear(title=True, xlabel=True, ylabel=True)
\end{verbatim}
Method for removing the title and/or xlabel and/or Ylabel
		of a matplotlib Plot.  Only to be called in an ipython 
		environment\newline
		input:\newline
		Title -  boolean of if title will be cleared \newline
		xlabel - boolean of if xlabel will be cleared \newline
		ylabel - boolean of if ylabel will be cleared \newline\newline

\begin{verbatim}
xlimrev()
\end{verbatim}
Reverses the xrange of a matplotlib Plot.\newline\newline
\begin{verbatim}
sparse(x,y,sparse)
\end{verbatim}
Method that removes every non sparse th element.  For example 
if this argument was 5, This method would plotthe 0th, 5th, 10th ... elements.\newline
Input:\newline
x: list of x values, of lenthe j\newline
y: list of y values, of lenthe j\newline
sparse: Argument that skips every so many data points\newline\newline

\subsection{utils.py}
The current iteration currenly only has the class variables elements\_names and stable\_el.
They are, respectively, a list of element names and stable elements for mppnp.py.

\section{Extention and Modification}

\subsection{Extention}

One goal behind the creation of these superclasses, is that these classes should be able to extend any module, thus giving it all the methods and variables contained in 
data\_plot.py and utils.py.  To do this there are some requirements:\newline
A. Place 'from data\_table import *' or 'from utils import *' at the top of the module\newline
B. If the class is defined like 'class MyClass:', change that to 
   'class MyClass(DataTable):' or 'class MyClass(Utils):'\newline
C. To properly use DataTable's methods properly one will need these methods:
	a get(atri) that returns a numpy array of Data, or a 
	list of numpy arrays of data.  The arguments of this function would need to be
	atri which is the name of the data one is looking for.\newline\newline

\subsection{Modification}
To Do
\section{History} 
This document history complements the svn log.

\begin{tabular*}{\textwidth}{lll}
\hline
Authors & yymmdd & Comment \\
\hline
DAC & 101115 & generate template and input documentation\\

\hline
\end{tabular*}
\subsection{To Do}
-Finish modification subsection.


\section{Contact}
If any bugs do appear or if there are any questions, please email hoshi@uvic.ca
% --------------- latex template below ---------------------------



%\begin{figure}[htbp]
%   \centering
%%   \includegraphics[width=\textwidth]{layers.jpg} % 
%      \caption{}   \includegraphics[width=0.48\textwidth]{FIGURES/HRD90ms.png}  
%   \includegraphics[width=0.48\textwidth]{FIGURES/HRD150ms.png}  
%
%   \label{fig:one}
%\end{figure}
%
%\begin{equation}
%Y\_a = Y\_k + \sum\_{i \neq k} Y\_i
%\end{equation}
%
%{
%%\color{ForestGreen}
%\sffamily 
%  {\center  --------------- \hfill {\bf START: Some special text} \hfill ---------------}\\
%$Y\_c$ does not contain ZZZ but we may assign one $Y_n$ to XYZ which is the decay product of the unstable nitrogen isotope JJHJ. %
%
%{\center ---------------  \hfill {\bf END:Some special text} \hfill ---------------}\\
%}
